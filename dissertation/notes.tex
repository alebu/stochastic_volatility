\documentclass [11pt,a4paper,oneside,openany]{book}
\usepackage{lmodern}
\usepackage[T1]{fontenc}
\usepackage[utf8]{inputenc}
\usepackage{cite}
\usepackage{graphicx} %Pacchetto necessario per la gestione delle immagini
\usepackage{listings} %Per inserire codice
\usepackage[usenames]{color} %Per permettere la colorazione dei caratteri
\usepackage{subfigure}
\usepackage{amsmath} %Simboli matematici
\usepackage{amsthm}

\usepackage{amsfonts} %Simboli matematici
\usepackage{amssymb} %Simboli matematici
%%\usepackage{Pacchetti/mcode}  %Pacchetto per l'immissione di codice matlab
%%\lstloadlanguages{C}
\usepackage {fancyhdr} %Pacchetto per la gestione accurata della pagina
\usepackage {setspace} %Pacchetto necessario per i comandi successivi onehalfspacing, singlespacing....
\usepackage{longtable} %Pacchetto per la gestione delle tabelle grandi
\usepackage[colorlinks=true]{hyperref} %Segnalibri nel pdf finale, sotto relativi parametri
\usepackage{eurosym}
\hypersetup{
	bookmarksnumbered=true,
	linkcolor=black,
	citecolor=black,
	urlcolor=black,
}
\lstset{
	language=C,
	basicstyle=\small\ttfamily,
	keywordstyle=\color{green}\bfseries,
	commentstyle=\color{red},
	stringstyle=\color{blue},
	showstringspaces=false
}
\usepackage{geometry} % Dimensione pagina
\geometry{a4paper} % Formato carta
\addtolength{\textheight}{65pt} %Margini
\oddsidemargin 30pt

\begin{document} %Inizio Documento

%%FRONTESPIZIO%%=========================================
\begin{titlepage}
%  \begin{center}
%      \includegraphics[width=6cm]{Immagini/logo.png}\\ %Logo dell'Università, cambiare il percorso, se necessario
%      \vspace{1em}
%      {\Large \textsc{Universit\`a degli studi di Perugia}}\\
%      \vspace{1em}
%      {\Large \textsc{Facolt\`a di Ingegneria}}\\
%      \vspace{5em}
%      {\LARGE \textbf{Appunti di Sicurezza Informatica}}\\
%  \end{center}
 
\vskip 2.5cm
\begin{center}
{\normalsize Anno Accademico 2014/2015}
\end{center}
\end{titlepage}

%%INTESTAZIONI PAGINA%%====================================
\pagestyle{fancy}
\renewcommand{\chaptermark} [1]{\chaptername\ \thechapter.\ #1}{} 
\renewcommand{\chaptermark}[1]{\markboth{\thechapter.\ #1}{}} 
\renewcommand{\sectionmark}[1]{\markright{\thesection\ #1}}
\fancyhf{}
\fancyhead[LE,RO]{\bfseries\thepage} 
\fancyhead[LO,RE]{\bfseries\leftmark} 
\fancypagestyle{plain}{%
\fancyhead{} % toglie l'intestazione
\renewcommand{\headrulewidth}{0pt} % e la linea
}

\frontmatter
%%INDICE%%==============================================
\tableofcontents
\mainmatter

%%CAPITOLI===============================================
\chapter{The need for extending Black-Scholes-Merton}

\section{Notation and preliminary results}

Let's start by introducing some notation and stating some preliminary results that will be useful at later stages.

\section{Stylized Facts}

\subsection{Volatility Smiles}

\section{Importance for Hedging}


\subsubsection{Crashofobia}


%%BIBLIOGRAFIA============================================
\normalsize
\newpage
\bibliography{bibliography}
\bibliographystyle{plain}

\end{document}